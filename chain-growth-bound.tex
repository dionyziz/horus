\section{Auxiliary Theorems}

The following theorem establishes that chains cannot grow too quickly. It uses the
notation adopted from the backbone series~\cite{backbone,varbackbone}.

\begin{theorem}[Chain Growth Bound]
  In a typical execution,
consider a round $r_0$ during which the longest chain that exists on the network has
a height of $h_0$. Then at round $r_1 > r_0$, let $h_1$ denote the height of the longest
chain that exists on the network. The chain cannot grow too quickly:

\[
  \Pr[h_1 - h_0 > (1 + \epsilon) (r_1 - r_0)nq \frac{T}{2^\kappa}] \leq \exp(-\Omega(\kappa (r_1 - r_0)))
\]
\end{theorem}
\begin{proof}
  Let us consider the case where the adversary uses all her queries (if the adversary does not use
  all of her queries, we can force her to do so at the end of her round and ignore the results).
  Then there will be $nq$ queries per round in total, and $(r_1 - r_0)nq$ queries across all the rounds
  in $r_1 - r_0$. In typical executions, the longest chain on the network can only grow if a query is successful.
  The probability of success of a query is $\frac{T}{2^\kappa}$. The random variable $h_1 - h_0$ is
  hence upper bounded by a Binomial distribution with parameters $\frac{T}{2^\kappa}$ and $(r_1 - r_0)nq$,
  which has expectation $(r_1 - r_0)nq \frac{T}{2^\kappa}$. Applying a Chernoff bound with error $\epsilon$,
  we obtain the desired result.
\end{proof}

We include the Chernoff bound, referenced at a high level throughout this paper, for completeness.

\begin{theorem}[Chernoff bounds]
  Let $\{X_i:i\in[n]\}$ are mutually independent Boolean random variables,
  with $\Pr[X_i=1]=p$, for all $i\in[n]$. Let $X=\sum_{i=1}^nX_i$ and $\mu=pn$.
  Then, for any $\delta\in(0,1]$,
  \[\Pr[X\le(1-\delta)\mu]\le e^{-\delta^2\mu/2}\hbox{~and~}
    \Pr[X\ge(1+\delta)\mu]\le e^{-\delta^2\mu/3}.\]
\end{theorem}
