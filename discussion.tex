\section{Discussion}

Having completed the presentation of our schemes, we now discuss a couple of
shortcomings and advantages of our scheme that differentiate it from previously
known solutions.

\noindent
\textbf{A bounty for the miners.}
In our analysis, we have considered a rational adversary who is only allowed
to allocate her capital into taking guesses for the user passwords and OTPs
and holds a minority of the adversarial power. This worldview is
slightly myopic. An adversary with a large capital operating in an open world
can also use this money for other purposes such as bribing miners.
In fact, let us
take a step back and reconsider the
\emph{honest majority} assumption of the chain which allowed us to conclude
that the properties of common prefix, safety, and liveness hold. What if the
miners are not honest, but rational, instead? In this case, the properties
do not hold (it is known that the honest protocol is not a Nash
equilibrium~\cite{selfish}, although it may be close to it~\cite{mining-games}).
In our case of keyless wallets, the wallet functions as a \emph{bounty}
to the miner who creates a long chain fork: If an
adversary can violate the common prefix property, then she can, as far
as the chain is concerned ``go back in time.'' In such an attack,
after the secrets become timelock decryptable, the adversary creates a
long chain reorganization and resubmits the correct guess to the wallet.
As the reorganization was long, the delay check in the smart contract
succeeds and the trial is correctly committed to the chain,
granting the adversary the prize. This can be dangerous.

However, It can be argued that such bounties can be created by the adversary
herself: If she double spends her money, she creates an incentive
for herself to go back in time and reclaim it. But there is
a crucial difference: The adversary can only spend her own money in the double spending
case. Namely, although in a double spending case, the party receiving the
money was harmed by the chargeback, it is the adversary's money that is
being double spent, not someone else's. There is another critical
difference here: While a single adversary can create such bounties for
herself, keyless wallets are \emph{universal} bounties claimable by any miner.
We remark, however, that a double spending adversary
can twist the double spending attack in a way that makes it a universal
bounty: The adversary first
creates a legitimate transaction spending some of her own money and
receives, say, fiat money in exchange. She then creates a double spending
transaction in an alternative fork: That double spending transaction
pays $50\%$ of her money to herself, and the rest to the miner who
confirms the given block. In this case, all miners are incentivized
to confirm this transaction and fork.

Therefore, we argue that the existing blockchain systems are not very
different in the way incentives are aligned as compared to our proposed
wallet. Nevertheless, as highlighted in the analysis, the proposed wallet
does indeed have a different security model from a standard wallet: It
is not purely cryptographic, but a cryptographic/cryptoeconomic hybrid.
A quantified analysis of the (ir)rationality of conducting a common prefix
attack is explored by Bonneau et al.~\cite{bonneau2015bitcoin}.

\noindent
\textbf{Temporary dishonest majority.}
Our analysis assumed adversarial minority
throughout the execution. However, this may not necessarily be the case.
Blockchains have faced situations where
the adversarial power has temporary majority spikes,
even though the adversary generally controls only a
minority~\cite{temporary-dishonest,supremacy}.
One of the arguments protecting from double spending stems from
the ability of the user to set their own local $k$ parameter when they
consider which transaction to accept as confirmed. The parameter $k$ is
not a global parameter of the system, but it can be set by each user
individually at the time of payment. If there are rumours or evidence
that the chain may be under attack, the user can delay accepting payments.
This is not the case for our protocol. While the user can set the critical
$2k$ delay when instantiating the contract, and this is a local choice,
this choice cannot be changed later. If an adversary attains majority
after the contract has been instantiated, she will be able to roll
back the chain sufficiently to steal the user's money. This limitation
of the system must be taken into account when deciding about the $k$
parameter of the wallet: The parameter must not only withstand current
adversarial bounds, but adversarial bounds through the lifetime of the
wallet. One mechanism to deal with this issue is to migrate from a
wallet with a small $k$ to a different wallet with a more
conservative value when evidence appears that the chain may become
attacked in the near future. If the blockchain network cannot be
trusted to maintain honest majority, and the adversarial majority spike
length is unpredictable, the money cannot be left and forgotten
as in a key-based wallet.

\noindent
\textbf{Detectability of brute force attacks.}
One of the advantages of our system is that brute force attacks are
not only economically infeasible, but they are also detectable. If an
adversary submits a brute force attempt to the wallet, a \emph{commit}
transaction will appear on the chain that the honest user will see.
As such, the user can decide to move their funds out of their wallet
if they observe such behavior. This benefit stems from the fact that
the brute forcing of user passwords and OTPs cannot be done offline,
but must necessarily be made on the chain. Our rate limiting scheme
is therefore not just enabling limiting, but also detection. It is
the first scheme of its kind that works in a decentralized manner
on-chain.
