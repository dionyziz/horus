\section{Introduction}

The management of cryptocurrency~\cite{bitcoin} wallet
private keys is a hassle.
Can we get rid of them and replace them with a simple short password or
a rotating $6$-digit one-time password (OTP)~\cite{rfc6238,rfc4226}? Users are more familiar with this
model, but this seems, at first sight, impossible to achieve:
The blockchain is public infrastructure, and anyone has access to the
public keys and smart contracts~\cite{buterin,wood} governing the conditions under which
one can spend. Any such short password will be easily broken by an offline
brute force attack~\cite{brain-drain}.

Perhaps unexpectedly, it \emph{is} possible to build brute force resilient wallets
by leveraging the blockchain infrastructure itself.
We build the first \emph{keyless cryptocurrency wallets}.
We propose two constructions. Our first construction is a password-based wallet.
It operates as follows.
Initially, Alice inputs a short (perhaps $6$ alphanumeric characters long) secret password into
the newly installed wallet software, as well as a \emph{maturity date}. The wallet software creates
a smart contract wallet containing the encrypted password on
the blockchain and outputs a wallet address to which money can be deposited at any time.
The money is unspendable prior to the maturity date. At any time prior to the maturity date, Alice
can open up a newly installed wallet instance and input the public wallet address and the
secret password. The newly installed wallet does not hold any secret information such as
private keys. Alice also enters the target address to which she wishes the money to go.
The wallet broadcasts two transactions on the blockchain: A \emph{commit} transaction immediately,
and a \emph{reveal} transaction later, on the maturity date. The money is securely transferred to
the target address. The wallet becomes unusable after the maturity date. This construction
is a stepping stone for the next.

Our second and final construction, the \emph{Hours of Horus},
is an OTP-based wallet. It operates as follows.
Alice initially uses her mobile wallet to generate a high entropy OTP seed.
This seed is used to generate
a large amount of time-based OTPs (with, say, hourly resolution) which are encrypted
and collected into a Merkle tree. The wallet creates a smart contract containing the
Merkle tree root on the blockchain and outputs a wallet address to which money can
be deposited at any time. The internal nodes of the Merkle tree are posted on a public
high availability location such as IPFS~\cite{ipfs} and can also be kept by Alice
in any untrusted device, if desired for availability. Alice then disconnects the mobile wallet and keeps it
completely airgapped and offline. At any time, Alice can use the offline device to generate
a time-based OTP. Without plugging in the offline device via USB or connecting it
to the Internet, Alice visually copies the short (perhaps $6$ alphanumeric characters long)
OTP that appears on the device's screen into her online computer. The wallet on
her online computer can then be used to input a target address and amount to be transferred.
Contrary to our previous construction, this second construction allows the user
to spend money \emph{at any time} (for the price of maintaining
the offline OTP device). As the OTPs are very short, this wallet is highly usable.
After the initial OTP seed generation, the secrets are kept in an
airgapped device, ensuring any bugs in the hardware or software cannot be abused
to steal keys or secrets.

Critical to the security of both constructions is ensuring that no adversary can
brute force the short password or OTPs.
Towards that goal, we devise a new cryptographic mechanism to secure cryptocurrency
wallet passwords from offline brute forcing attempts. Any adversary who wishes
to brute force these passwords \emph{must} do so through the chain itself and
record the attempt in a transaction. As such, these attempts are governed by
the limitations of the chain: Each transaction costs gas to perform. This
gives rise to the first \emph{decentralized rate limiting} mechanism. Through
appropriate cryptoeconomic parametrization, we ensure that the adversary will,
in expectation, and with any desired probability, lose much more money than they
will win out of brute forcing attempts. The parametrization dictates the length
of the password based on current transaction gas costs and the capital to be
protected.

To achieve this property, we leverage the fact that the network is performing
proof-of-work~\cite{pow} (or proof-of-stake~\cite{ouroboros}) in a predictable rate in expectation~\cite{bitcoin-sok}. We use \emph{Witness Encryption}
(WE)~\cite{STOC:GGSW13} to encrypt
the password in such a way that it can only be decrypted using the \emph{future}
proof-of-work that will be performed by the network. As such, the encryption
is a \emph{Timelock Encryption}~\cite{timelock} in which the miners function in tandem~\cite{timelock-bitcoin} to
decrypt the submitted password. This decryption is a by-product of
the proof-of-work they are performing anyway. The miners do not need to know that the
passwords have been timelock encrypted. The security of timelock encryption
ensures that the passwords will not be decryptable prior to the chain progressing
a certain number of blocks. Our security argument stands upon five pillars:

\begin{enumerate}
      \item a secure \emph{extractable} Witness Encryption scheme,
      \item a secure underlying blockchain (with \emph{Common Prefix}),
      \item a secure hash function (modelled as a \emph{Random Oracle}),
      \item a secure \emph{pseudorandom} OTP scheme, and
      \item a rational, or limited capital, adversary.
\end{enumerate}

Our constructions could, in principle, be deployed to any smart-contract--enabled
proof-of-work chain such as Ethereum. In particular, we do not require any modifications
to the Ethereum consensus mechanism or smart contract virtual machine. The
best known instantiation of the Witness Encryption primitive, which the Timelock Encryption
instance makes use of, requires the use of \emph{multilinear maps}.
Multilinear maps are (approximately) constructible using ideal lattices.
Unfortunately, this construction currently remains impractical.
Until such constructions are built, our scheme is of theoretical interest.

\noindent
\textbf{Our contributions.}
The contributions of this paper are summarized as follows:

\begin{itemize}
  \item We introduce a timelock-based \emph{keyless} cryptocurrency wallet in which funds are spent by
        just using a short password at a prespecified \emph{maturity date}.
  \item We introduce the first timelock-based high usability \emph{OTP} wallet, with OTP length of
        just 6 alphanumeric characters. The funds can be spent at any time just by
        providing the OTP password from an airgapped device.
  \item In the process of building these wallets, we put forth the first
        \emph{decentralized rate limiting} scheme. The scheme protects the user
        from brute force attacks by an adversary, by requiring all attempts to
        be recorded on the chain.
\end{itemize}

\noindent
\textbf{Related work.}
Witness Encryption was introduced by Garg et al.~\cite{STOC:GGSW13} (based on the \textsc{Exact
Cover} problem). In that
work, they propose a construction which makes use of lattice-based approximate multilinear
maps proposed in previous work~\cite{EC:GarGenHal13}. The lattice-based approach has been
improved~\cite{EC:LanSteSte14} and its implementation details further explored~\cite{AC:ACLL15}.
A series of attacks on this construction have been discovered~\cite{EC:CHLRS15,EC:HuJia16,PKC:CLLT17,C:CGHLMM15,C:AlbBaiDuc16,EPRINT:CheJeoLee16}.
A follow-up lattice-based approach~\cite{TCC:GenGorHal15} was put forth later, but also attacked~\cite{C:CLLT16}.
An integer-based implementation of multilinear maps has also been proposed~\cite{C:CorLepTib13,C:CorLepTib15} and
attacked~\cite{EC:CHLRS15,EPRINT:MinFou15,EPRINT:CheLeeRyu15}.
Current advancements~\cite{ma2018mmap} seem, so far, immune to such attacks.
All in all, the state of the art consists of constructions that have poorly
understood security and sometimes exotic cryptographic assumptions.

Timelock Encryption was introduced by Rivest et al.~\cite{timelock}. A Timelock Encryption
making use of Witness Encryption (based on the \textsc{Subset Sum} problem) and blockchain proof-of-work was proposed by Liu et
al.~\cite{timelock-bitcoin}. This is the first instance of timelock encryption in which
a computationally limited decryptor can rely on the blockchain miners for the decryption
of secrets. We make heavy use of their scheme in this work. Applications of this scheme to cryptocurrencies
have been previously discussed by Miller~\cite{miller-timelock}.

A taxonomy of cryptocurrency wallets and their security is put forth by Karantias~\cite{wallet-taxonomy}.
The security of offline wallets is explored by Karakostas et al.~\cite{hardware-wallets}.
The use of OTP~\cite{rfc6238,rfc4226} as a mechanism for cryptocurrency wallet protection has been explored in SmartOTP~\cite{smartotp}.
The SmartOTP work was a great inspiration for the present paper. Contrary to our construction, their
wallet uses OTPs as a second factor (in addition to private keys) and requires large-length OTP strings
which makes it less usable, but their implementation is quite practical and even has a Solidity implementation.
Password-based cryptocurrency wallets have previously appeared as \emph{brain wallets},
but brute force attacks against them have proven catastrophic~\cite{brain-drain}.
