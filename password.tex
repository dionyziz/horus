\section{A Password Wallet}\label{sec:password}

Let us start by building a password-based wallet without the use of private keys.
This construction will be a stepping stone for the next. In this construction, we
will have a severe limitation: The wallet can only be used to spend \emph{once},
and at a \emph{predetermined time}. Once the wallet has been used, it cannot be
reused with the same password. Furthermore, the wallet becomes unusuable if the
funds have not been spent prior to the maturity date.

From a user point of view, the wallet works as follows. Initially, Alice chooses
a secret password with $\lambda$ bits of entropy. We will determine this $\lambda$
later, but let us say, with foresight, that it will be enough to have a password
just $6$ alphanumeric characters long. Alice also chooses a \emph{maturity date},
a timestamp in the future (expressed as chain height),
and uses her wallet software to generate
a smart contract which she then posts on the chain. This generates a public wallet
address for Alice that she can use to receive multiple payments prior to the maturiy date.
The wallet software can then
be discarded and no secret information needs to be kept by Alice, beyond the
secret password that she remembers, and the public contract that remains on the chain.
No private keys are stored anywhere.
A short period before the maturity date arrives,
Alice uses the wallet software to connect to the chain network, and enters her
password and desired destination. The software issues two transactions to the chain:
First, a $\textsf{tx}_\textsf{commit}$ transaction, which lets Alice illustrate prior
knowledge of her secret password; and, second, a $\textsf{tx}_\textsf{reveal}$ transaction,
in which Alice proves that her previous commitment indeed corresponds to the secret
password committed on the chain. The first transaction is posted strictly prior to the maturity
date, while the second transaction is posted on or after the maturity date.

\import{./}{algorithms/alg.password.tex}

Let us explore how the contract is implemented.
The smart contract construction for the wallet is illustrated in
Algorithm~\ref{alg.password}. It consists of three methods: A \textsf{construct}
method, called when the wallet is initialized; a \textsf{commit} method, called
shortly prior to the maturity date; and a \textsf{reveal} method, called after
the maturity date. These two last methods are used for spending.

The interaction with the wallet is illustrated in Algorithm~\ref{alg.password.external}.
When Alice wishes to deploy her wallet, she
begins by generating a password $sk \getsrandomly \{0, 1\}^\lambda$. She also chooses
a future timestamp at which she will be able to spend her money.
She submits this information to her
software wallet.
The software wallet connects to
the blockchain and observes the current stable blockchain tip $B = \chain[-k]$
and its height $t_0$.
Alice's timestamp choice is translated to a future block height $\Delta \in \mathbb{N}$ which
denotes how far in the future,
in block height after $t_0$, she wants to spend her money: If $\Delta = 100$, the money will
be spendable when the blockchain reaches height $t_0 + \Delta$ blocks.
We set $t_1 = t_0 + \Delta$ to be the block height at which spending becomes possible.
The software wallet constructs the contract of Algorithm~\ref{alg.password}
by broadcasting its construction transaction $\textsf{tx}_\textsf{construct}$
to the blockchain network.
The constructor accepts two parameters: The $t_1$ parameter,
and the $c$ parameter. The $c$ parameter is a timelock-encrypted
ciphertext of her password. Concretely, Alice's software wallet
sets $c = \textsf{timelock}(sk, t)$ by invoking $\textsf{WE.Enc}_\mathcal{R}(sk, x)$.
Here, $\mathcal{R}$ denotes the polynomially computable relation validating block headers
starting from a particular block and continuing up to a prespecified block height, checking
ancestry and proof-of-work nonces, as described in the preliminaries. The problem instance $x = (B, t)$
is the tuple consisting of the latest known stable block and the maturity height.
Observe now that the ciphertext $c$ which is published on the
smart contract and known to the adversary is a ciphertext which can only be decrypted
after $t$ blocks have been mined on top of block $B$. The transaction returns a wallet address $pk$
at which she can receive money prior to the maturity height.

To spend her money, Alice runs the wallet software anew and inputs her public wallet address $pk$,
her password $sk$, and destination address $\alpha_{\textsf{to}}$. The wallet software does not have any information beyond this.
The software runs an honest chain node which observes a chain $\chain$.
At any time before its local chain reaches height $t_1 - \ell - 2k$,
the wallet generates a new high-entropy salt $\textsf{salt} \getsrandomly \{0, 1\}^\kappa$ (where $\kappa$
is a security parameter in the order of $128$). This salt
is short-lived ($\ell - 2k$ blocks) and must survive until the chain reaches height $t_1$.
It then creates a transaction $\textsf{tx}_\textsf{commit}$.
This transaction contains
a commitment $z$ evaluated as $z = H(\left<sk, \textsf{salt}, \alpha_{\textsf{to}}\right>)$,
where we assume that the encoding $\left<\cdot\right>$ denotes a uniquely decodable encoding
of the triplet.
This transaction is submitted to the smart contract by invoking the \textsf{commit} method.
Due to the liveness of the ledger, the transaction is confirmed in a block with
height at most $t_1 - 2k$. The smart contract records the commitment, as the requirement
in Line~\ref{alg.password.delay} is satisfied, and stores it in the \textsf{commitments} set.

After the local chain of the wallet reaches height $t_1$ (due to the Chain Growth lower bound, this
will be soon enough~\cite{backbone}), the software gathers the block headers
$\chain[t_0{:}t_1]$ to construct a timelock witness $w$. It then creates a transaction
$\textsf{tx}_\textsf{reveal}$ which invokes the \textsf{reveal} method of the smart contract
and includes the plaintext password $sk$, which now becomes public, the plaintext
$\textsf{salt}$, which also becomes public information, the target address $\alpha_{\textsf{to}}$, and
the witness $w$. The \textsf{reveal} method checks that the submitted data corresponds to the
previous commitment, and that the stored encrypted password $c$ timelock decrypts to the provided
password $sk$. In that case, it forwards the money to the $\alpha_{\textsf{to}}$ address.

\import{./}{algorithms/alg.password.external.tex}

We now give a high-level overview of the correctness and security of this scheme.
The \emph{correctness} property of the wallet mandates that the honest wallet user can
create a valid spending transaction, i.e., a transaction which executes
\emph{reveal} to completion. The \emph{security} property mandates that the adversary
cannot create a valid transaction. These properties, together, ensure that the honest user
can spend her money, while the adversary cannot.

On the one hand, the scheme is correct, because the honest user can always create the commit
and reveal transactions in order, and, due to liveness, these cannot be censored. When time
$t_1$ arrives, the smart contract can verify the veracity of the claims and issue the final
transfer. On the other hand, the scheme is secure, because, prior to time $t_1$ the adversary
does not hold a chain of length $t_1$. Without such a chain, the adversary cannot distinguish
a correct from an incorrect guess, due to the security of witness encryption. Any guess the
adversary makes is a good as any. However, all of these guesses must be committed to the smart
contract sufficiently before time $t_1$ arrives. And, so, the adversary must choose to blindly
submit some guesses and hope that some of them are correct. This very soon becomes uneconomical,
even if the password length is quite short. We give more details on the security of the scheme
in the subsequent Analysis section.

One detail to note here is that the time $t_1$ is given in block time. Because the rate of
blockchain growth can vary, the honest party must monitor the chain and ensure that block
height $t_1$ has not passed. This is one additional limitation of this scheme that makes it
unusable. Additionally, the password can only be used once.
In the next section, we lift these limitations.
