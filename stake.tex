\section{Proof-of-Stake}

We now adapt our OTP wallet construction to the proof-of-stake setting (the password wallet can also
be adapted likewise). For concreteness, we describe
a construction for the \emph{Ouroboros}~\cite{ouroboros} and \emph{Ouroboros Praos}~\cite{praos}
systems, but our results are extensible to other systems as well (such as Snow
White~\cite{DBLP:journals/iacr/BentovPS16a} and Algorand~\cite{algorand}).

The construction does not change much from the proof-of-work case, so we only provide a sketch
of the construction here. The smart contract remains identical, except
for the moderately hard \textsc{NP} language describing the existence of a proof-of-work witness.
More concretely, the problem instance $x$ is now $(\rho, sl, D, t)$, where $\rho$ denotes the randomness
of the current epoch, $sl$ denotes the
slot during which block $B$ (the most recently known stable block $\chain[-k]$) was generated,
$D$ denotes the stake distribution during the current epoch, and $t$ denotes
the future time.
While the proof-of-stake chains also enjoy the common prefix property,
unfortunately, we cannot simply take any blockchain that has length $t$ following $B$,
because the adversary can create blockchains of arbitrary length. The proof-of-stake system
ensures that such chains are not taken into account by checking that any blockchain
received on the network does not contain blocks that were issued in future slots~\cite{ouroboros}.
However, we cannot incorporate this check in the form of an \textsc{NP} language, as
we do not have access to a clock.

Instead, we rely on a critical property of the proof-of-stake system that states that,
in any consecutive $2k$ slots, at least $k+1$ will be honestly allocated.
Therefore, we reinterpret the parameter $t$ to mean the \emph{number of slots} after
block $B$ instead of the \emph{number of blocks}. The witness $w$ consists
of two parts: \emph{block} data and \emph{epoch} data.
The block data contains a sequence
$(\sigma_1, H_1, \pi_1, sl_1),\allowbreak (\sigma_2, H_2, \pi_2, sl_2),\allowbreak
 \cdots,\allowbreak (\sigma_d, H_d, \pi_d, sl_d)$
of signatures
$\sigma_i$ each with their corresponding slot $sl_i$, with $sl_i > sl$ and
a proof of leadership $\pi_i$. As in the proof-of-work case, no transaction data is verified.
The epoch data contains a sequence
$(\rho_1, u_1),\allowbreak (\rho_2, u_2),\allowbreak \cdots,\allowbreak (\rho_e, u_e)$
spanning all the epochs starting from the epoch of slot $sl$ up until the epoch
of slot $sl + t$. For each of these, the randomness $\rho_j$ and evicence $u_j$ of the multiparty
computation leading to it (typically a collection of signatures) is included.

The relation $\mathcal{R}$ polynomially verifies that all the signatures $\sigma_i$
correctly sign their respective plaintext $H_i$, that
the proofs of leadership $\pi_i$ are correct, and that the evidence $u_j$ for the
randomness $\rho_j$ of each epoch is correct.
Critically, it also checks that, for every window of length $2k$ slots, at least
$k+1$ blocks have been provided.

This completes the basic scheme. We can improve upon this scheme by noting
that blocks and signatures for everything but the most recent epoch
are not necessary, as long as the randomness and its evidence for each epoch is
given. This evidence can be made quite short using ATMs schemes, in which the
evidence consists of aggregate threshold signatures
(c.f., ~\cite{pos-sidechains}). In such an optimization, a constant amount of bits
is required per epoch. Blocks only need to be presented for the last
epoch in order to have better time granularity. However, here, too, some pruning can
occur: It is sufficient that only $k+1$
blocks and signatures pertaining to the most recent $2k$ slots of the most recent epoch
are presented.
The relation $\mathcal{R}$ can then simply check the evidence for each epoch randomness,
that the $k+1$ signatures are correct, that they all fall within a $2k$ window, and
that the slot during which the last such block was generated is $t$.
Again, the witness encryption can be composed with a zk-SNARK
to make the witnesses constant size.

Contrary to proof-of-work where the velocity of the chain is unknown, despite bounded,
in the proof-of-stake case we have a much better grasp on how quickly the time $t$
will be reached, as it is a slot number. While the adversary still enjoys some early
leakage ($k$ slots early), the timelocked data will be available at the prespecified time.
In the proof-of-work case, it is possible that the blockchain growth rate will
increase or decrease due to the stochastic nature of block production. As such, the
proof-of-stake scheme is naturally fitting to the timelock problem.
