\section{Conclusion}

We presented the first wallets that work securely without private
keys, developing a wallet in which the user spends with an
OTP from an air gapped device. We proved our scheme secure
through a hybrid cryptographic/cryptoeconomic argument which may be of
independent interest (in the appendix). The cryptoeconomic analysis led
to a short OTP code: Just $6$ alphanumeric characters suffice
even for large capital of seven figures and with a conservative economic
margin of $90\%$ capital loss for the adversary. Our calculations
were also conservative with respect to fees.

We extended our scheme to work in the proof-of-stake model, as well
as variable difficulty proof-of-work model (in the appendix).
We are the first to extend timelock encryption to proof-of-stake and to effectively
use it for the variable proof-of-work case as well (previous
considerations~\cite{timelock-bitcoin} considered the variable difficulty
case, but did not account for the fact that the decryption time will be
varying with the miner population adjustments).

As far as we know, our work is the first to build any useful protocol,
and certainly to construct wallets, on top of timelock encryption and
blockchains. We believe that timelock encryption and witness encryption
is a promising cryptographic direction and, once established, will prove
to be cornerstones of future protocol development for blockchains.
