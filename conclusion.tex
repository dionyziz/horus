\section{Conclusion}

We have presented the first wallets that work securely without any private
keys. We developed a password-based wallet with severe limitations: It could
only be used once and at a predetermined maturity date. We then improved upon
our first scheme by proposing an OTP wallet in which the user can provide an
OTP from an air gapped device. We analyzed our scheme and proved it secure
through a hybrid cryptographic/cryptoeconomic argument which may be of
independent interest. The cryptoeconomic parametrization allowed us to
set the OTP code to be quite short: Just $6$ alphanumeric characters sufficed
even for large capital of seven figures and with a very conservative economic
margin of $90\%$ capital loss for the adversary in expectation. Our calculations
were also conservative with respect to current fees.

We finally extended our scheme to work in the proof-of-stake model, as well
as the variable difficulty proof-of-work model. While timelock encryption
using blockchain witnesses leveraging witness encryption has been described
before, we are the first to extend it to proof-of-stake and to effectively
use it for the variable proof-of-work case as well (previous
considerations~\cite{timelock-bitcoin} considered the variable difficulty
case, but did not account for the fact that the decryption time will be
varying with the miner population adjustments).

As far as we know, our work is the first to build any useful protocol,
and certainly to construct wallets, on top of timelock encryption and
blockchains. We believe that timelock encryption and witness encryption
is a promising cryptographic direction and, once established, will prove
to be cornerstones of future protocol development for blockchains (and
elsewhere). However, it remains to be seen whether a secure instantiation of
these primitives will ever become practical.
