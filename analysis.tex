\section{Analysis}\label{sec:analysis}

We now give a more complete analysis of the scheme. First, let us prove that the Password-based
wallet of Algorithm~\ref{alg.password} is correct.

\begin{theorem}[Password Wallet Correctness (Informal)]
  Let the blockchain have \emph{liveness} and \emph{safety},
  and let the witness encryption scheme $\textsf{WE}$ be \emph{correct}.
  An honest party spending at block height $t_1 - \ell - 2k$ or earlier
  will generate a valid spending transaction
  for Algorithm~\ref{alg.password}.
\end{theorem}
\begin{proof}[Sketch]
  The contract is created when $B = \chain[-k]$ is stable. Due to safety, all the future chains
  will be extending this block. The contract is initialized with $x = (B, t)$ by issuing the
  $\textsf{tx}_\textsf{construct}$ transaction. Due to liveness, this transaction is confirmed
  within $\ell$ blocks.
  The honest user then creates a transaction $\textsf{tx}_\textsf{commit}$ when her
  own chain has length $|\chain| = t_1 - \ell - 2k$.
  Due to liveness, this transaction becomes confirmed for all honest parties after $\ell$ blocks
  have elapsed, and is placed in position $\chain[t_1 - 2k]$ or earlier. Therefore, Line~\ref{alg.password.delay}
  of the method \emph{commit} succeeds. When $|\chain| = t_1$, the honest user calls
  \emph{reveal}, passing $w$. Due to the correctness of the witness encryption scheme,
  the decryption succeeds. The password and salt revealed match the ones committed.
  Due to liveness, this transaction becomes confirmed.
\end{proof}

The correctness of the OTP-based scheme is similar.

\begin{theorem}[OTP Wallet Correctness (Informal)]
  Let the blockchain have \emph{liveness} and \emph{safety},
  and let the witness encryption scheme $\textsf{WE}$ be \emph{correct}.
  An honest party spending multiple times prior to $\textsf{MAX\_TIME} - \ell - 2k$
  will succeed in creating valid transactions.
  in Algorithm~\ref{alg.otp}.
\end{theorem}
\begin{proof}[Sketch]
  The proof is the same as above, with the difference that the value $t$ is provided
  at the \textsf{commit} time, not the \textsf{construct} time. The argument that
  Line~\ref{alg.otp.delay} will be successful remains the same due to liveness.
\end{proof}

Our security analysis is in a hybrid \emph{cryptographic} and \emph{cryptoeconomic}
setting.
In the system described, we have two security parameters. First, we have
the \emph{cryptographic} security
parameter $\kappa$ ($\approx 128$ bits), which determines the security of the hash function,
the security of the witness encryption scheme, and the security of the blockchain (in terms
of liveness, safety, and common prefix). The probability of failure is negligible in this
parameter. Any breakage in this parameter can be catastrophic for the system and
can potentially provide the adversary with gains without any cost.
Secondly, we have the much shorter \emph{cryptoeconomic} security parameter $\lambda$
($\approx 35$ bits) which denotes the entropy of the chosen user password $sk$ or the
length of each OTP $\textsf{OTP}_t$.
While this parameter is hopelessly short from a cryptographic
point of view (and $2^{-35}$ is nothing but negligible), we will use it to establish
a lower bound in the economic cost of an attack. In particular, we will tweak this
parameter so that the return-on-investment of an attack can be made arbitrarily close
to $-100\%$. The result will be that the adversary can make the probability of
success non-negligible, but at an economic cost which renders such attempts irrational.

We begin by stating our Decentralized Rate Limiting lemma, which establishes that
an adversary must necessarily submit transactions to the blockchain in order to
have any non-negligible probability of success. The probability of success is
determined by the number of transactions submitted by the adversary and made
persistent by the system. Based on this result, we will determine the cryptoeconomic
parametrization ($\lambda$) required to make the system economically infeasible to attack.

\begin{lemma}[Decentralized Rate Limiting (Informal)]\label{lem:rate-limit}
  Consider a static difficulty proof-of-work
  blockchain with \emph{safety} and \emph{common prefix}.
  Let the hash function $H$ be collision-resistant and preimage-resistant, and let
  the witness encryption scheme $\textsf{WE}$ be a \emph{secure witness encryption with witness extractability}.
  A PPT adversary who submits fewer than $g$ transactions that are
  eventually confirmed by all honest parties has a probability of
  achieving a valid spending transaction in Algorithm~\ref{alg.password}
  upper bounded by $\frac{g}{2^\lambda} + \textsf{negl}(\kappa)$.
\end{lemma}
\begin{proof}[Sketch]
  In order for the adversary to have a valid transaction, she must have created
  a $\textsf{tx}_\textsf{reveal}$ in which she passes a password $sk$, a \textsf{salt} and
  a $\alpha'_{\textsf{to}}$ address which is different from the honestly provided $\alpha_{\textsf{to}}$ address.
  This \textsf{reveal} transaction must be confirmed into the chain $\chain$ adopted by a verifier honest
  party $P_v$ and have matching data with a previous $\textsf{tx}_\textsf{commit}'$ transaction which
  was placed earlier in $\chain$. Additionally, $\textsf{tx}_\textsf{commit}'$ must be in $\chain[t_1 - 2k]$
  or earlier (due to the check in Line~\ref{alg.password.delay}).
  Due to the collision resistance of $H$, the respective commit transaction must be different
  from the one ($\textsf{tx}_\textsf{commit}$) provided by the honest party, as
  $\alpha_{\textsf{to}} \neq \alpha'_{\textsf{to}}$.

  Let $r_c$ denote the round during which the spender honest party $P_s$ broadcasts
  their $\textsf{tx}_\textsf{commit}$ transaction to the network, and let $r_z$
  denote the last round during which \emph{all} honest parties have chains with length
  of at most $t_1 - k$.

  Let us consider all adversarially generated commit transactions $\textsf{tx}^i_\textsf{commit}$ ($i \geq 1$)
  that are eventually reported as \emph{stable} by $P_v$
  (the adversary can also create transactions that do not make it in the chain of $P_v$,
  but we will not count these). For these transactions, let us consider the round $r_i$ during which
  each of these transactions $\textsf{tx}^i_\textsf{commit}$ was created.

  \textbf{Case 1:} $r_i < r_c$.
  Since the honest spender has not yet submitted a commitment,
  the only information that the adversary has is the ciphertext $c$. If at this round the
  adversary can distinguish between $sk$ and any other plaintext in $\{0, 1\}^\lambda$
  with probability non-negligible in $\kappa$, then, due to the witness extractability of
  \textsf{WE}, an extractor can extract a witness $w$ attesting to the existence of a chain
  of height $t_1$. But in that case, we can perform a computational reduction to an
  adversary that breaks the common prefix property of the chain by producing a chain of
  height $t_1$ at round $r_i$ when the honest party $P_v$ has adopted a chain of length
  only $t_1 - \ell - 2k$. This breaks the common prefix assumption.

  \textbf{Case 2:} $r_c \leq r_i \leq r_z$.
  In this case, the honest spender has broadcast a commitment to the network, but there
  are no chains of length $t_1$. The adversary now holds both the timelocked ciphertext $c$
  and the commitment $z$. Again the adversary should not be able to distinguish between $sk$
  and any other plaintext in $\{0, 1\}^\lambda$, except with probability negligible in
  $\kappa$ (recall that the $\textsf{salt}$ is kept secret and has $\kappa$ bits of entropy).
  Otherwise, we can either perform a reduction to a common-prefix-breaking adversary
  making use of witness extractability, or we can perform a reduction to a
  preimage-resistance-breaking adversary.

  \textbf{Case 3:} $r_i > r_z$.
  By the definition of $r_z$, in round $r_i$ there must exist an honest party
  with a chain of length at least $t_1 - k$. By the common prefix property, all
  other honest parties have a chain of length exceeding $t_1 - 2k$.

  Let us consider what happens in all of these three cases. In the first two cases, any
  \emph{single} guess that the adversary places into a transaction can be correct with
  probability $\frac{1}{2^\lambda} + \textsf{negl}(\kappa)$. In the third
  case, while the adversary can potentially guess with better probability (due to the chain
  reaching its leakage point $t_1 - k$), any such transactions can never make it into
  the chain eventually adopted by $P_v$, as the check in Line~\ref{alg.password.delay}
  will fail.

  As the transactions that eventually make it into the chain of $P_v$ were all generated prior
  to $r_z$, the probability that each of them is a valid spending transaction is
  upper bounded by $\frac{1}{2^\lambda} + \textsf{negl}(\kappa)$.
  If the adversary submits at most $g$
  such transactions, and applying a union bound, the overall probability of success
  is $g(\frac{1}{2^\lambda} + \textsf{negl}(\kappa)) = \frac{g}{2^\lambda} + \textsf{negl}(\kappa)$.
\end{proof}

The above Lemma is identical for our other construction. We state it for completeness.

\begin{lemma}[OTP Decentralized Rate Limiting (Informal)]\label{lem:rate-limit-otp}
  Consider a static difficulty proof-of-work
  blockchain with \emph{safety} and \emph{common prefix}.
  Let the hash function $H$ be collision-resistant and preimage-resistant, and let
  the witness encryption scheme $\textsf{WE}$ be a \emph{secure witness encryption with witness extractability}.
  Let $\mathcal{G}$ be a secure pseudorandom function
  $\{0, 1\}^\kappa \times \mathbb{N} \longrightarrow \{0, 1\}^\lambda$.
  A PPT adversary who submits fewer than $g$ transactions that are
  eventually confirmed by all honest parties has a probability of
  achieving a valid spending transaction in Algorithm~\ref{alg.otp}
  upper bounded by $\frac{g}{2^\lambda} + \textsf{negl}(\kappa)$.
\end{lemma}
\begin{proof}[Sketch]
  The proof here is identical to Lemma~\ref{lem:rate-limit},
  noting that each of the different $\textsf{OTP}_t$
  essentially gives rise to independent attack paths to the adversary.
  Due to the pseudorandom nature of the OTP scheme, any previous OTPs
  do not reveal any information to our polynomial adversary.
  An adversary making a spending attempt has
  a probability of $\frac{1}{2^\lambda}$ of succeeding in each attempt,
  unless it can be reduced to a collision-resistance-breaking adversary,
  a common-prefix-breaking adversary, or an adversary breaking the
  pseudorandomness of the OTP scheme. But all of these events are negligible
  in $\kappa$.
\end{proof}

At this point, we have established that the probability of success is negligible
in both parameters $\kappa$ and $\lambda$. However, we will keep the parameter $\lambda$
short, and we will make the $\kappa$ parameter reasonably long ($\kappa = 128$).
Setting $\lambda = \kappa$ would, of course, give sufficient security. The reason
for separating these two parameters is that the $\lambda$ parameter affects the usability
of the system: It is the number of characters that must be remembered by the user
in the case of a password, or the number of characters that must be visually copied
by the user in the case of an OTP.

In the above result, we have expressed the probability as a sum of two terms:
$\frac{g}{1^\lambda} + \textsf{negl}(\kappa)$. This reflects the nature of the
two parameters: We opt to calculate the \emph{concrete} probability with respect
to $\lambda$, but only give an asymptotic probability with respect to $\kappa$.
This treatment hints at our intentions: Our high-level argument was to condition
the system to the overwhelming events that there will be no cryptographic breakage
in the hash function, common prefix property, blockchain safety, blockchain liveness,
and OTP pseudorandomness. Conditioned under these events, the concrete probability as
a function of $\lambda$
allows us to make an argument of why any attack is uneconomical. This
gives rise to our (cryptoeconomic) security theorem.

\begin{theorem}[Cryptoeconomic Security (Informal)]\label{thm:security}
  Consider a chain with fee $f$ per transaction. If the wallet of
  Algorithm~\ref{alg.password} or Algorithm~\ref{alg.otp} is used with a maximum capital of $V$,
  then the parametrization $\lambda > \log \frac{V}{f}$
  yields a
  negative expectation of income for the adversary, with overwhelming probability in $\kappa$.
  Additionally,
  the expected return-on-investment for this adversary is at most
  $\frac{V}{f 1^\lambda} - 1$.
\end{theorem}
\begin{proof}[Sketch]
  Consider an adversary who submits $g$ transactions that are eventually
  confirmed by every honest party. This adversary is irrevocably investing a
  capital of $gf$ for this attack. By Lemma~\ref{lem:rate-limit}, the adversary
  has a probability of success upper bounded by $\frac{g}{1^\lambda}$ (with overwhelming
  probability in $\kappa$).
  The expected income for this adversary is at most
  $\Ex[\textsf{income}] \leq V \frac{g}{1^\lambda} - gf$.
  Taking $\lambda > \log \frac{V}{f}$, we obtain
  $\Ex[\textsf{income}] < 0$. The expected return-on-investment is
  $\frac{\Ex[\textsf{income}]}{gf} - 1$.
\end{proof}

In this scheme, we can set $\lambda$ big enough to make the return-on-investment as close to $-100\%$ as we want.
If we want the return-on-investment to be $-1 + \epsilon$ for some $\epsilon \in (0, 1]$, we let
$\lambda = \log \frac{V}{f \epsilon}$. In short, we can make the adversary lose an amount arbitrarily close to all their money
in expectation.

To consider some concrete parametrization of the scheme, let us assume that we wish to establish a
target $-90\%$ ($\epsilon = 0.1$) expected return-on-investment for the adversary in a wallet where we want to store up to
$V = \$100{,}000$ in capital at any point in time. Consider a blockchain where the fees per transaction
are\footnote{This price corresponds to Ethereum--fiat prices and gas fees for
simple transfer transactions in May 2021. As smart contract transactions are significantly
more expensive, this is a conservative estimation for the fees.} at least $f = \$1.60$. We obtain
$\lambda = \log \frac{V}{f \epsilon} = \log_2 625{,}000 < 20$ bits. This corresponds to just $6$
numerical characters (base $10$), or just $4$ alphanumeric characters (base $58$).
A standard OTP authenticator such as Google's Authenticator application is therefore
appropriate for such parameters.
Increasing
the maximum capital that will be stored in the wallet by three orders of magnitude to
$\$100{,}000{,}000$ requires $5$ alphanumeric characters instead.
